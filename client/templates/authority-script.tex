\documentclass{a4form}
\usepackage{multicol}
\usepackage{graphicx}
% PBS authority script form

\DeclareRobustCommand{\checkbox}{\framebox(3,3){$\bullet$}}
\DeclareRobustCommand{\emptycheckbox}{\framebox(3,3){}}
\DeclareRobustCommand{\text}[4]{\put(#1,-#2){
\parbox[t]{#3 mm}{#4}
}}
\renewcommand{\normalsize}{\fontsize{9pt}{9pt}\selectfont}
\begin{document}
\begin{page}

%\begin{center}
\text{5}{3}{100}{\textbf{!authority-heading-authority-script-number}}
%\end{center}
%\begin{center}
\text{110}{3}{100}{\textbf{!authority-heading-authority-script-number}}
%\end{center}
\text{5}{12}{80}{!doctorname \\ !practitioner-address-vertical  Phone:!clinic-phone}
\text{25}{27}{80}{!prescriber-no}
%\text{70}{27}{80}{script number !script-number}
\text{110}{12}{80}{!doctorname \\  !practitioner-address-vertical  Phone:!clinic-phone}
\text{130}{27}{80}{!prescriber-no}
%\text{160}{27}{80}{script number !script-number}
\text{30}{58}{80}{!patientname \\ !patient-address-vertical}
\text{140}{58}{80}{!patientname \\ !patient-address-vertical}
\text{10}{73}{80}{!script-date}
\text{125}{73}{80}{!script-date}
\text{70}{73}{80}{Send to patient !sendtopatient-checkbox}
\text{160}{73}{80}{Send to patient !sendtopatient-checkbox}
\text{10}{79}{20}{\textbf{!pbs-script}}
\text{30}{79}{20}{\textbf{!rpbs-script}}
\text{120}{79}{20}{\textbf{!pbs-script}}
\text{120}{79}{20}{\textbf{!rpbs-script}}
\text{32}{34}{80}{!medicare}
\text{135}{34}{80}{!medicare}
\lineh{22}{83}{80}
\lineh{128}{83}{80}
\text{22}{87}{80}{\normalsize !drugs}
\text{128}{87}{80}{\normalsize !drugs}
\lineh{22}{100}{80} 
\lineh{128}{100}{80} 
\text{22}{105}{80}{There is one item listed on this script}
\text{128}{105}{80}{There is one item listed on this script}
\text{22}{120}{80}{Signature of !doctorname...........................}
\text{128}{120}{80}{Signature of !doctorname...........................}
\text{22}{130}{80}{Authority Approval Number !authority-approval-number}
\text{128}{130}{80}{Authority Approval Number !authority-approval-number}
\text{22}{135}{80}{Quantity !quantity-repeats}
\text{128}{135}{80}{Quantity !quantity-repeats}
\text{22}{150}{80}{Authorized:}
\text{127}{150}{80}{Authorized:}
\text{22}{154}{80}{Delegate  :}
\text{127}{154}{80}{Delegate  :}
\text{5}{195}{95}{\textbf{MEDICARE AUSTRALIA/DVA COPY   !script-date\\AUTHORITY NO. !authority-script-number}}
\text{105}{195}{95}{\textbf{DOCTORS COPY   !script-date\\AUTHORITY NO. !authority-script-number}}
\text{5}{202}{80}{!doctorname \\ !practitioner-address-vertical !clinic-phone  PN !prescriber-no}
\text{105}{202}{80}{!doctorname \\  !practitioner-address-vertical !clinic-phone  PN !prescriber-no}
\lineh{6}{212}{90}     % Authority tear off left side horizontal line  under dr & practice details
\lineh{106}{212}{90}     % Authority tear off left side horizontal line  under dr & practice details
\text{5}{215}{80}{!patientname \\ !patient-address-vertical}
\text{105}{215}{80}{!patientname \\ !patient-address-vertical}
\lineh{6}{223}{90}     % Authority tear off left side horizontal line  under dr & practice details
\lineh{106}{223}{90}     % Authority tear off left side horizontal line  under dr & practice details
\text{5}{226}{80}{\normalsize !drugs}
\text{105}{226}{80}{\normalsize !drugs}
\lineh{6}{239}{90}     % Authority tear off right side horizontal line  under dr & practice details
\lineh{106}{239}{90}     % Authority tear off left side horizontal line  under dr & practice details
\text{5}{243}{80}{Previous Authority? Yes !previous-authority-yes-checkbox No !previous-authority-no-checkbox}
\text{105}{243}{80}{Previous Authority? Yes !previous-authority-yes-checkbox No !previous-authority-no-checkbox }
\text{70}{73}{80}{Send to patient !sendtopatient-checkbox}
\text{5}{247}{90}{\textbf{INDICATION FOR USE OF ITEM-}!authority-indication}
\text{105}{247}{90}{\textbf{INDICATION FOR USE OF ITEM-}!authority-indication}



\end{page}
\end{document}
