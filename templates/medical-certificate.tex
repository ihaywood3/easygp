\documentclass[a4paper,12pt]{article}
\usepackage{multicol}
\usepackage{graphicx}
%------------------------------------------------------------------------------------------
% EasyGP Pregnancy Dates Form work in progress - still just a copy of a med certificate
% Don't panic if this .text template won't print if you try it useing pdfLatex for example
% in a terminal. It will crash on the option button lines, but then again, its meant to
% as the values for these are set in code!
% i.e !temporality_is, !temporality_was, !temporality_will_be, fit_for_work, unfit_for_work
%------------------------------------------------------------------------------------------
%\renewcommand{\normalsize}{\fontsize{12pt}{12pt}\selectfont}
\setlength{\hoffset}{-1in}
\setlength{\voffset}{-1in}
\setlength{\parindent}{0pt}
\setlength{\textwidth}{210mm}
\setlength{\textheight}{295mm}
\renewcommand{\unitlength}{mm}
\DeclareRobustCommand{\lineh}[3]{\put(#1,-#2){\line(1,0){#3}}}
\DeclareRobustCommand{\linev}[3]{\put(#1,-#2){\line(0,-1){#3}}}
\DeclareRobustCommand{\text}[4]{\put(#1,-#2){ \parbox[t]{#3 mm}{#4}}}
\DeclareRobustCommand{\checkbox}{\framebox(3,3){$\bullet$}}
\DeclareRobustCommand{\emptycheckbox}{\framebox(3,3){}}

\begin{document}
\begin{picture}(210,295)(0,-295)
\text{60}{-10}{220}{
\textbf{\normalsize  MEDICAL CERTIFICATE}}

\text{3}{10}{60}{
\textbf{\footnotesize !clinic-name}\\
\footnotesize !clinic-branch \\
\footnotesize !clinic-street \\
\footnotesize !clinic-suburb\\
\\
\footnotesize Phone: !clinic-phone\\
\footnotesize Fax  : !clinic-fax\\
}


\text{130}{10}{60}{
\textbf{\footnotesize !practitioner-name}\\
\footnotesize !provider-no }


\lineh{3}{40}{180}  % horizontal line, under the practice heading, above the patient's details

\text{3}{45}{55}{
\textbf{\footnotesize Patient Details}\\
\footnotesize !name \\
\footnotesize !street \\
\footnotesize !suburb \\}

\lineh{3}{60}{180}  % horizontal line, below the patient's details

\text{3}{85}{180}{
This is to certify that the above patient has been issued with this medical certificate by our practice 
regarding their fitness for work for periods stated. \scriptsize (See privacy footnote below)}

\text{3}{100}{180}{
\normalsize The patient  }

\text{40}{100}{20}{
!temporality_is {\normalsize Is}
}
\text{60}{100}{20}{
!temporality_was {\normalsize Was}
}
\text{80}{100}{20}{
!temporality_will_be {\normalsize Will be}
}

\text{40}{110}{20}{
!fit_for_work {\normalsize Fit}
}

\text{60}{110}{30}{
!unfit_for_work {\normalsize UnFit}
}

\text{40}{120}{150}{
\normalsize  For work from  !from_date  to   !to_date}

\text{40}{130}{150}{
\normalsize    !additional_notes}

\text{3}{155}{55}{
\textbf{\normalsize Yours Sincerely}
}
\text{3}{185}{55}{
\textbf{\footnotesize Date:  }
\normalsize !certificate.date }

\text{3}{230}{180}{
\scriptsize Footnote: The requirement for a person to produce a Medical Certificate pursuant to his or her absence
from work in New South Wales is regulated by the particular Award under which the employee works.
Medical practitioners may not disclose to any third-party information they acquired during the course of the
professional relationship. The nature of the patient's illness is confidential information and is not required on
medical certificates unless (i) the patient consents; (ii) the certificate is seen only by another medical
practitioner; and (iii) confidentiality is guaranteed. It is illegal to back-date certificates.}
                 
\text{57}{260}{80}{\tiny EasyGP version 0.1r}

\end{picture}
\end{document}
